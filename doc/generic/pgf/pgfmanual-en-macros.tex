% Copyright 2019 by Till Tantau
%
% This file may be distributed and/or modified
%
% 1. under the LaTeX Project Public License and/or
% 2. under the GNU Free Documentation License.
%
% See the file doc/generic/pgf/licenses/LICENSE for more details.

% $Header$


\newcount\pgfmanualtargetcount

\colorlet{examplefill}{yellow!80!black}
\definecolor{graphicbackground}{rgb}{0.96,0.96,0.8}
\definecolor{codebackground}{rgb}{0.9,0.9,1}
\definecolor{animationgraphicbackground}{rgb}{0.96,0.96,0.8}




\makeatletter
\newenvironment{pgfmanualentry}%
{%\LWR@restoreoriglists% % Dominik: this was necessary on my TeXLive 2021 installation, but not necessary with TeXLive 2022.
\BlockClass{manualentry}%
\list{}{\leftmargin=2em\itemindent-\leftmargin\def\makelabel##1{\hss##1}}}%
{\endlist%
\endBlockClass}%
\makeatother
\newcounter{pgfmanualentry}
\newcommand\pgfmanualentryheadline[1]{%
\begin{BlockClass}{entryheadline}
  \itemsep=0pt\parskip=0pt{\raggedright\item\refstepcounter{pgfmanualentry}\strut{#1}\par}\topsep=0pt%
\end{BlockClass}}
\newcommand\pgfmanualbody{\parskip3pt}

\let\origtexttt=\texttt
\def\texttt#1{{\def\textunderscore{\char`\_}\def\textbraceleft{\char`\{}\def\textbraceright{\char`\}}\origtexttt{#1}}}
\def\exclamationmarktext{!}
\def\atmarktext{@}

{
  \catcode`\|=12
  \gdef\pgfmanualnormalbar{|}
  \catcode`\|=13
  \AtBeginDocument{\gdef|{\ifmmode\pgfmanualnormalbar\else\expandafter\verb\expandafter|\fi}}
}



\newenvironment{pgflayout}[1]{
  \begin{pgfmanualentry}
    \pgfmanualentryheadline{%
      \pgfmanualpdflabel{#1}{}%
      \definitionspan{\texttt{\string\pgfpagesuselayout\char`\{\declare{#1}\char`\}}\oarg{options}}%
    }
    \index{#1@\protect\texttt{#1} layout}%
    \index{Page layouts!#1@\protect\texttt{#1}}%
    \pgfmanualbody
}
{
  \end{pgfmanualentry}
}


\newenvironment{sysanimateattribute}[1]{
  \begin{pgfmanualentry}
    \pgfmanualentryheadline{%
      \pgfmanualpdflabel{#1}{}%
      \definitionspan{\texttt{\string\pgfsysanimate\char`\{\declare{#1}\char`\}}}%
    }
    \index{#1@\protect\texttt{#1} system layer animation attribute}%
    \index{Animation attributes (system layer)!#1@\protect\texttt{#1}}%
    \pgfmanualbody
}
{
  \end{pgfmanualentry}
}


\newenvironment{animateattribute}[1]{
  \begin{pgfmanualentry}
    \pgfmanualentryheadline{%
      \pgfmanualpdflabel{#1}{}%
      \definitionspan{\texttt{\string\pgfanimateattribute\char`\{\declare{#1}\char`\}\marg{options}}}%
    }
    \index{#1@\protect\texttt{#1} basic layer animation attribute}%
    \index{Animation attributes (basic layer)!#1@\protect\texttt{#1}}%
    \pgfmanualbody
}
{
  \end{pgfmanualentry}
}


\newenvironment{tikzanimateattribute}[1]{
  \begin{pgfmanualentry}
    \pgfmanualentryheadline{%
      \foreach \attr in{#1} {\expandafter\pgfmanualpdflabel\expandafter{\attr}{}}%
      \definitionspan{\textbf{Animation attribute} \foreach \attr[count=\i]
      in{#1}{{\ifnum\i>1 \textbf,\fi} \texttt{:\declare{\attr}}}}%
    }
    \foreach\attr in{#1}{%
      \edef\indexcall{%
        \noexpand\index{\attr@\noexpand\protect\noexpand\texttt{\attr} animation attribute}%
        \noexpand\index{Animation attributes!\attr@\noexpand\protect\noexpand\texttt{\attr}}%
      }%
      \indexcall%
    }%
    \pgfmanualbody
}
{
  \end{pgfmanualentry}
}


\newenvironment{command}[1]{
  \begin{pgfmanualentry}
    \extractcommand#1\@@
    \pgfmanualbody
}
{
  \end{pgfmanualentry}
}

\makeatletter

\def\includeluadocumentationof#1{
  \directlua{require 'pgf.manual.DocumentParser'}
  \directlua{pgf.manual.DocumentParser.include '#1'}
}

\newenvironment{luageneric}[4]{
  \pgfmanualentry
    \pgfmanualentryheadline{\definitionspan{#4 \texttt{#1\declare{#2}}#3}}
    \index{#2@\protect\texttt{#2} (Lua)}%
    \def\temp{#1}
    \ifx\temp\pgfutil@empty\else
      \index{#1@\protect\texttt{#1}!#2@\protect\texttt{#2} (Lua)}%
    \fi
  \pgfmanualbody
}{\endpgfmanualentry}

\newenvironment{luatable}[3]{
  \medskip
  \luageneric{#1}{#2}{ (declared in \texttt{#3})}{\textbf{Lua table}}
}{\endluageneric}

\newenvironment{luafield}[1]{
  \pgfmanualentry
    \pgfmanualentryheadline{\definitionspan{Field \texttt{\declare{#1}}}}
  \pgfmanualbody
}{\endpgfmanualentry}


\newenvironment{lualibrary}[1]{
  \pgfmanualentry
  \pgfmanualentryheadline{%
    \pgfmanualpdflabel{#1}{}%
    \definitionspan{\textbf{Graph Drawing Library} \texttt{\declare{#1}}}%
  }
    \index{#1@\protect\texttt{#1} graph drawing library}%
    \index{Libraries!#1@\protect\texttt{#1}}%
    \index{Graph drawing libraries!#1@\protect\texttt{#1}}%
    \vskip.25em
    \texttt{\char`\\usegdlibrary\char`\{\declare{#1}\char`\}\space\space \char`\%\space\space  \LaTeX\space and plain \TeX}\\
    \texttt{\char`\\usegdlibrary[\declare{#1}]\space \char`\%\space\space Con\TeX t}\smallskip\par
    \pgfmanualbody
}{\endpgfmanualentry}

\newenvironment{luadeclare}[4]{
  \pgfmanualentry
  \def\manual@temp@default{#3}%
  \def\manual@temp@initial{#4}%
  \def\manual@temp@{#3#4}%
  \pgfmanualentryheadline{%
    \pgfmanualpdflabel{#1}{}%
    \definitionspan{\texttt{/graph
      drawing/\declare{#1}\opt{=}}\opt{#2}}\hfill%
    \ifx\manual@temp@\pgfutil@empty\else%
    (\ifx\manual@temp@default\pgfutil@empty\else%
    default \texttt{#3}\ifx\manual@temp@initial\pgfutil@empty\else, \fi%
    \fi%
    \ifx\manual@temp@initial\pgfutil@empty\else%
    initially \texttt{#4}%
    \fi%
    )\fi%
  }%
  \index{#1@\protect\texttt{#1} key}%
  \pgfmanualbody
  \gdef\myname{#1}%
%  \keyalias{tikz}
%  \keyalias{tikz/graphs}
}{\endpgfmanualentry}

\newenvironment{luadeclarestyle}[4]{
  \pgfmanualentry
  \def\manual@temp@para{#2}%
  \def\manual@temp@default{#3}%
  \def\manual@temp@initial{#4}%
  \def\manual@temp@{#3#4}%
  \pgfmanualentryheadline{%
    \pgfmanualpdflabel{#1}{}%
    \definitionspan{\texttt{/graph drawing/\declare{#1}}\ifx\manual@temp@para\pgfutil@empty\else\opt{\texttt=}\opt{#2}\fi}\hfill%
    (style\ifx\manual@temp@\pgfutil@empty\else, %
    \ifx\manual@temp@default\pgfutil@empty\else%
    default \texttt{#3}\ifx\manual@temp@initial\pgfutil@empty\else, \fi%
    \fi%
    \ifx\manual@temp@initial\pgfutil@empty\else%
    initially \texttt{#4}%
    \fi%
    \fi)%
  }%
  \index{#1@\protect\texttt{#1} key}%
  \pgfmanualbody%
  \gdef\myname{#1}%
%  \keyalias{tikz}
%  \keyalias{tikz/graphs}
}{\endpgfmanualentry}

\newenvironment{luanamespace}[2]{
  \luageneric{#1}{#2}{}{\textbf{Lua namespace}}
}{\endluageneric}

\newenvironment{luafiledescription}[1]{}{}

\newenvironment{luacommand}[4]{
  \hypertarget{pgf/lua/#1}{}\luageneric{#2}{#3}{\texttt{(#4)}}{\texttt{function}}
}{\endluageneric}

\newenvironment{luaparameters}{\par\emph{Parameters:}%
  \parametercount=0\relax%
  \let\item=\parameteritem%
  \let\list=\restorelist%
}
{\par
}

\newenvironment{luareturns}{\par\emph{Returns:}%
  \parametercount=0\relax%
  \let\item=\parameteritem%
  \let\list=\restorelist%
}
{\par
}

\newcount\parametercount

\newenvironment{parameterdescription}{\unskip%
  \parametercount=0\relax%
  \let\item=\parameteritem%
  \let\list=\restorelist%
}
{\par
}
\let\saveditemcommand=\item
\let\savedlistcommand=\list
\def\denselist#1#2{\savedlistcommand{#1}{#2}\parskip0pt\itemsep0pt}
\def\restorelist{\let\item=\saveditemcommand\denselist}
\def\parameteritem{\pgfutil@ifnextchar[\parameteritem@{}}%}
\def\parameteritem@[#1]{\advance\parametercount by1\relax\hskip0.15em plus 1em\emph{\the\parametercount.}\kern1ex\def\test{#1}\ifx\test\pgfutil@empty\else#1\kern.5em\fi}

\newenvironment{commandlist}[1]{%
  \begin{pgfmanualentry}
  \foreach \xx in {#1} {%
    \expandafter\extractcommand\xx\@@
  }%
  \pgfmanualbody
}{%
  \end{pgfmanualentry}
}%

% \begin{internallist}[register]{\pgf@xa}
% \end{internallist}
%
% \begin{internallist}[register]{\pgf@xa,\pgf@xb}
% \end{internallist}
\newenvironment{internallist}[2][register]{%
  \begin{pgfmanualentry}
  \foreach \xx in {#2} {%
    \expandafter\extractinternalcommand\expandafter{\xx}{#1}%
  }%
  \pgfmanualbody
}{%
  \end{pgfmanualentry}
}%
\def\extractinternalcommand#1#2{%
  \removeats{#1}%
  \pgfmanualentryheadline{%
    \pgfmanualpdflabel{\textbackslash\strippedat}{}%
    \definitionspan{Internal #2 \declare{\texttt{\string#1}}}}%
  \index{Internals!\strippedat @\protect\myprintocmmand{\strippedat}}%
  \index{\strippedat @\protect\myprintocmmand{\strippedat}}%
}

%% MW: START MATH MACROS
\def\mvar#1{{\ifmmode\textrm{\textit{#1}}\else\rmfamily\textit{#1}\fi}}

\makeatletter

\def\extractmathfunctionname#1{\extractmathfunctionname@#1(,)\tmpa\tmpb}
\def\extractmathfunctionname@#1(#2)#3\tmpb{\def\mathname{#1}}

\makeatother
  
\newenvironment{math-function}[1]{
  \def\mathdefaultname{#1}
  \extractmathfunctionname{#1}
  \edef\mathurl{{math:\mathname}}\expandafter\hypertarget\expandafter{\mathurl}{}%
  \begin{pgfmanualentry}
    \pgfmanualentryheadline{\definitionspan{\texttt{#1}}}%
    \index{\mathname @\protect\texttt{\mathname} math function}%
    \index{Math functions!\mathname @\protect\texttt{\mathname}}%
    \pgfmanualbody
}
{
  \end{pgfmanualentry}
}

\def\pgfmanualemptytext{}
\def\pgfmanualvbarvbar{\char`\|\char`\|}

\newenvironment{math-operator}[4][]{%
  \begin{pgfmanualentry}
  \csname math#3operator\endcsname{#2}{#4}
  \def\mathtest{#4}%
  \ifx\mathtest\pgfmanualemptytext%
    \def\mathtype{(#3 operator)}
  \else%
    \def\mathtype{(#3 operator; uses the \texttt{#4} function)}
  \fi%
  \pgfmanualentryheadline{\definitionspan{\mathexample\qquad\mathtype}}%
  \def\mathtest{#1}%
  \ifx\mathtest\pgfmanualemptytext%
    \index{#2@\protect\texttt{#2} #3 math operator}%  
    \index{Math operators!#2@\protect\texttt{#2}}%
  \fi%
  \pgfmanualbody
}
{\end{pgfmanualentry}}

\newenvironment{math-operators}[5][]{%
  \begin{pgfmanualentry}
  \csname math#4operator\endcsname{#2}{#3}
  \def\mathtest{#5}%
  \ifx\mathtest\pgfmanualemptytext%
    \def\mathtype{(#4 operators)}
  \else%
    \def\mathtype{(#4 operators; use the \texttt{#5} function)}
  \fi%
  \pgfmanualentryheadline{\definitionspan{\mathexample\qquad\mathtype}}%
  \def\mathtest{#1}%
  \ifx\mathtest\pgfmanualemptytext%
    \index{#2#3@\protect\texttt{#2\protect\ #3} #4 math operators}% 
    \index{Math operators!#2#3@\protect\texttt{#2\protect\ #3}}%
  \fi%
  \pgfmanualbody
}
{\end{pgfmanualentry}}

\def\mathinfixoperator#1#2{%
  \def\mathoperator{\texttt{#1}}%
  \def\mathexample{\mvar{x}\space\texttt{#1}\space\mvar{y}}%
}

\def\mathprefixoperator#1#2{%
  \def\mathoperator{\texttt{#1}}%
  \def\mathexample{\texttt{#1}\mvar{x}}%
}

\def\mathpostfixoperator#1#2{%
  \def\mathoperator{\texttt{#1}}
  \def\mathexample{\mvar{x}\texttt{#1}}%
}

\def\mathgroupoperator#1#2{%
  \def\mathoperator{\texttt{#1\ #2}}%
  \def\mathexample{\texttt{#1}\mvar{x}\texttt{#2}}%
}

\expandafter\let\csname matharray accessoperator\endcsname=\mathgroupoperator
\expandafter\let\csname matharrayoperator\endcsname=\mathgroupoperator

\def\mathconditionaloperator#1#2{%
  \def\mathoperator{#1\space#2}
  \def\mathexample{\mvar{x}\ \texttt{#1}\ \mvar{y}\ {\texttt{#2}}\ \mvar{z}}
}

\newcommand\mathcommand[1][\mathdefaultname]{%
  \expandafter\makemathcommand#1(\empty)\stop%
  \expandafter\extractcommand\mathcommandname\@@%
  \medskip
}
\makeatletter

\def\makemathcommand#1(#2)#3\stop{%
  \expandafter\def\expandafter\mathcommandname\expandafter{\csname pgfmath#1\endcsname}%
  \ifx#2\empty%
  \else%
    \@makemathcommand#2,\stop,
  \fi}
\def\@makemathcommand#1,{%
  \ifx#1\stop%
  \else%
    \expandafter\def\expandafter\mathcommandname\expandafter{\mathcommandname\texttt{\char`\{#1\char`\}}}%
    \expandafter\@makemathcommand%
  \fi}
\makeatother

\def\calcname{\textsc{calc}}

\newenvironment{math-keyword}[1]{
  \extracttikzmathkeyword#1@
  \begin{pgfmanualentry}
    \pgfmanualentryheadline{\definitionspan{\texttt{\textcolor{red}{\mathname}}\mathrest}}%
    \index{\mathname @\protect\texttt{\mathname} tikz math function}%
    \index{TikZ math functions!\mathname @\protect\texttt{\mathname}}%
    \pgfmanualbody
}
{
  \end{pgfmanualentry}
}

\def\extracttikzmathkeyword#1#2@{%
  \def\mathname{#1}%
  \def\mathrest{#2}%
}

%% MW: END MATH MACROS


\def\extractcommand#1#2\@@{%
  \removeats{#1}%
  \pgfmanualentryheadline{%
    \pgfmanualpdflabel{\textbackslash\strippedat}{}%
    \definitionspan{\declare{\expandafter\texttt\expandafter{\string#1}}#2}%
  }%
  \index{\strippedat @\protect\myprintocmmand{\strippedat}}
}

\def\luaextractcommand#1#2\relax{%
  \declare{\texttt{\string#1}}#2\par%
%  \removeats{#1}%
 % \index{\strippedat @\protect\myprintocmmand{\strippedat}}
 % \pgfmanualpdflabel{\textbackslash\strippedat}{}%
}


% \begin{environment}{{name}\marg{arguments}}
\renewenvironment{environment}[1]{
  \begin{pgfmanualentry}
    \extractenvironement#1\@@
    \pgfmanualbody
}
{
  \end{pgfmanualentry}
}

\def\extractenvironement#1#2\@@{%
  \pgfmanualentryheadline{%
    \pgfmanualpdflabel{#1}{}%
    \definitionspan{\texttt{\char`\\begin\char`\{\declare{#1}\char`\}}#2}%
  }%
  \pgfmanualentryheadline{\definitionspan{\texttt{\ \ }\meta{environment contents}}}%
  \pgfmanualentryheadline{\definitionspan{\texttt{\char`\\end\char`\{\declare{#1}\char`\}}}}%
  \index{#1@\protect\texttt{#1} environment}%
  \index{Environments!#1@\protect\texttt{#1}}
}


\newenvironment{plainenvironment}[1]{
  \begin{pgfmanualentry}
    \extractplainenvironement#1\@@
    \pgfmanualbody
}
{
  \end{pgfmanualentry}
}

\def\extractplainenvironement#1#2\@@{%
  \pgfmanualentryheadline{\definitionspan{\texttt{\declare{\char`\\#1}}#2}}%
  \pgfmanualentryheadline{\definitionspan{\texttt{\ \ }\meta{environment contents}}}%
  \pgfmanualentryheadline{\definitionspan{\texttt{\declare{\char`\\end#1}}}}%
  \index{#1@\protect\texttt{#1} environment}%
  \index{Environments!#1@\protect\texttt{#1}}%
}


\newenvironment{contextenvironment}[1]{
  \begin{pgfmanualentry}
    \extractcontextenvironement#1\@@
    \pgfmanualbody
}
{
  \end{pgfmanualentry}
}

\def\extractcontextenvironement#1#2\@@{%
  \pgfmanualentryheadline{\definitionspan{\texttt{\declare{\char`\\start#1}}#2}}%
  \pgfmanualentryheadline{\definitionspan{\texttt{\ \ }\meta{environment contents}}}%
  \pgfmanualentryheadline{\definitionspan{\texttt{\declare{\char`\\stop#1}}}}%
  \index{#1@\protect\texttt{#1} environment}%
  \index{Environments!#1@\protect\texttt{#1}}}


\newenvironment{shape}[1]{
  \begin{pgfmanualentry}
    \pgfmanualentryheadline{%
      \pgfmanualpdflabel{#1}{}%
      \definitionspan{\textbf{Shape} \texttt{\InlineClass{keyname}{\declare{#1}}}}%
    }%
    \index{#1@\protect\texttt{#1} shape}%
    \index{Shapes!#1@\protect\texttt{#1}}
    \pgfmanualbody
}
{
  \end{pgfmanualentry}
}

\newenvironment{pictype}[2]{
  \begin{pgfmanualentry}
    \pgfmanualentryheadline{%
      \pgfmanualpdflabel{#1}{}%
      \definitionspan{\textbf{Pic type} \texttt{\declare{#1}#2}}%
    }%
    \index{#1@\protect\texttt{#1} pic type}%
    \index{Pic Types!#1@\protect\texttt{#1}}
    \pgfmanualbody
}
{
  \end{pgfmanualentry}
}

\newenvironment{shading}[1]{
  \begin{pgfmanualentry}
    \pgfmanualentryheadline{%
      \pgfmanualpdflabel{#1}{}%
      \definitionspan{\textbf{Shading} \texttt{\declare{#1}}}}%
    \index{#1@\protect\texttt{#1} shading}%
    \index{Shadings!#1@\protect\texttt{#1}}
    \pgfmanualbody
}
{
  \end{pgfmanualentry}
}


\newenvironment{graph}[1]{
  \begin{pgfmanualentry}
    \pgfmanualentryheadline{%
      \pgfmanualpdflabel{#1}{}%
      \definitionspan{\textbf{Graph} \texttt{\declare{#1}}}}%
    \index{#1@\protect\texttt{#1} graph}%
    \index{Graphs!#1@\protect\texttt{#1}}
    \pgfmanualbody
}
{
  \end{pgfmanualentry}
}

\newenvironment{gdalgorithm}[2]{
  \begin{pgfmanualentry}
    \pgfmanualentryheadline{%
      \pgfmanualpdflabel{#1}{}%
      \definitionspan{\textbf{Layout} \texttt{/graph drawing/\declare{#1}\opt{=}}\opt{\meta{options}}}}%
    \index{#1@\protect\texttt{#1} layout}%
    \index{Layouts!#1@\protect\texttt{#1}}%
    \foreach \algo in {#2}
    {\edef\marshal{\noexpand\index{#2@\noexpand\protect\noexpand\texttt{#2} algorithm}}\marshal}%
    \index{Graph drawing layouts!#1@\protect\texttt{#1}}
    \item{\small alias \texttt{/tikz/#1}}\par
    \item{\small alias \texttt{/tikz/graphs/#1}}\par
    \item{\small Employs \texttt{algorithm=#2}}\par
    \pgfmanualbody
}
{
  \end{pgfmanualentry}
}

\newenvironment{dataformat}[1]{
  \begin{pgfmanualentry}
    \pgfmanualentryheadline{%
      \pgfmanualpdflabel{#1}{}%
      \definitionspan{\textbf{Format} \texttt{\declare{#1}}}}%
    \index{#1@\protect\texttt{#1} format}%
    \index{Formats!#1@\protect\texttt{#1}}
    \pgfmanualbody
}
{
  \end{pgfmanualentry}
}

\newenvironment{stylesheet}[1]{
  \begin{pgfmanualentry}
    \pgfmanualentryheadline{%
      \pgfmanualpdflabel{#1}{}%
      \definitionspan{\textbf{Style sheet} \texttt{\declare{#1}}}}%
    \index{#1@\protect\texttt{#1} style sheet}%
    \index{Style sheets!#1@\protect\texttt{#1}}
    \pgfmanualbody
}
{
  \end{pgfmanualentry}
}

\newenvironment{handler}[1]{
  \begin{pgfmanualentry}
    \extracthandler#1\@nil%
    \pgfmanualbody
}
{
  \end{pgfmanualentry}
}

\def\gobble#1{}
\def\extracthandler#1#2\@nil{%
  \pgfmanualentryheadline{%
    \pgfmanualpdflabel{/handlers/#1}{}%
    \definitionspan{\textbf{Key handler} \meta{key}\texttt{/\declare{#1}}#2}}%
  \index{\gobble#1@\protect\texttt{#1} handler}%
  \index{Key handlers!#1@\protect\texttt{#1}}
}


\makeatletter


\newenvironment{stylekey}[1]{
\begin{pgfmanualentry}%
\def\extrakeytext{style, }%
\extractkey#1\@nil%
\pgfmanualbody%
}
{
\end{pgfmanualentry}
}

\def\choicesep{$\vert$}%
\def\choicearg#1{\texttt{#1}}

\newif\iffirstchoice

% \mchoice{choice1,choice2,choice3}
\newcommand\mchoice[1]{%
  \begingroup
  \firstchoicetrue
  \foreach \mchoice@ in {#1} {%
    \iffirstchoice
      \global\firstchoicefalse
    \else
      \choicesep
    \fi
    \choicearg{\mchoice@}%
  }%
  \endgroup
}%

% \begin{key}{/path/x=value}
% \begin{key}{/path/x=value (initially XXX)}
% \begin{key}{/path/x=value (default XXX)}
\newenvironment{key}[1]{
  \begin{pgfmanualentry}
    \def\extrakeytext{}
    %\def\altpath{\emph{\color{gray}or}}%
    \extractkey#1\@nil%
    \pgfmanualbody
}
{
  \end{pgfmanualentry}
}

% \insertpathifneeded{a key}{/pgf} -> assign mykey={/pgf/a key}
% \insertpathifneeded{/tikz/a key}{/pgf} -> assign mykey={/tikz/a key}
%
% #1: the key
% #2: a default path (or empty)
\def\insertpathifneeded#1#2{%
  \def\insertpathifneeded@@{#2}%
  \ifx\insertpathifneeded@@\empty
    \def\mykey{#1}%
  \else
    \insertpathifneeded@#2\@nil
    \ifpgfutil@in@
      \def\mykey{#2/#1}%
    \else
      \def\mykey{#1}%
    \fi
  \fi
}%
\def\insertpathifneeded@#1#2\@nil{%
  \def\insertpathifneeded@@{#1}%
  \def\insertpathifneeded@@@{/}%
  \ifx\insertpathifneeded@@\insertpathifneeded@@@
    \pgfutil@in@true
  \else
    \pgfutil@in@false
  \fi
}%

% \begin{keylist}[default path]
%   {/path/option 1=value,/path/option 2=value2}
% \end{keylist}
\newenvironment{keylist}[2][]{%
  \begin{pgfmanualentry}
    \def\extrakeytext{}%
  \foreach \xx in {#2} {%
    \expandafter\insertpathifneeded\expandafter{\xx}{#1}%
    \expandafter\extractkey\mykey\@nil%
  }%
  \pgfmanualbody
}{%
  \end{pgfmanualentry}
}%

\newcommand{\defaultspace}{\hspace{2em}}
\newcommand{\defaultspan}[1]{\InlineClass{hl-default}{#1}}
\newcommand{\definitionspan}[1]{\InlineClass{hl-def}{#1}}

\def\extractkey#1\@nil{%
  \pgfutil@in@={#1}%
  \ifpgfutil@in@%
    \extractkeyequal#1\@nil
  \else%
    \pgfutil@in@{(initial}{#1}%
    \ifpgfutil@in@%
      \extractequalinitial#1\@nil%
    \else
      \pgfmanualentryheadline{%
      \def\mykey{#1}%
      \def\mypath{}%
      \gdef\myname{}%
      \firsttimetrue%
      \pgfmanualdecomposecount=0\relax%
      \decompose#1/\nil%
        \definitionspan{\texttt{\InlineClass{keyname}{\declare{#1}}}}%
        \defaultspan{(\extrakeytext no value)}}%
    \fi
  \fi%
}

\def\extractkeyequal#1=#2\@nil{%
  \pgfutil@in@{(default}{#2}%
  \ifpgfutil@in@%
    \extractdefault{#1}#2\@nil%
  \else%
    \pgfutil@in@{(initial}{#2}%
    \ifpgfutil@in@%
      \extractinitial{#1}#2\@nil%
    \else
      \pgfmanualentryheadline{%
        \def\mykey{#1}%
        \def\mypath{}%
        \gdef\myname{}%
        \firsttimetrue%
        \pgfmanualdecomposecount=0\relax%
        \decompose#1/\nil%
        \definitionspan{\texttt{\InlineClass{keyname}{\declare{#1}}=}#2}%
        \defaultspan{(\extrakeytext no default)}}%
    \fi%
  \fi%
}

\def\extractdefault#1#2(default #3)\@nil{%
  \pgfmanualentryheadline{%
    \def\mykey{#1}%
    \def\mypath{}%
    \gdef\myname{}%
    \firsttimetrue%
    \pgfmanualdecomposecount=0\relax%
    \decompose#1/\nil%
    \definitionspan{\texttt{\InlineClass{keyname}{\declare{#1}}\opt{=}}\opt{#2}}%
    \defaultspan{(\extrakeytext default \texttt{#3})}}%
}

\def\extractinitial#1#2(initially #3)\@nil{%
  \pgfmanualentryheadline{%
    \def\mykey{#1}%
    \def\mypath{}%
    \gdef\myname{}%
    \firsttimetrue%
    \pgfmanualdecomposecount=0\relax%
    \decompose#1/\nil%
    \definitionspan{\texttt{\InlineClass{keyname}{\declare{#1}}=}#2}%
    \defaultspan{(\extrakeytext no default, initially \texttt{#3})}}%
}

\def\extractequalinitial#1 (initially #2)\@nil{%
  \pgfmanualentryheadline{%
    \def\mykey{#1}%
    \def\mypath{}%
    \gdef\myname{}%
    \firsttimetrue%
    \pgfmanualdecomposecount=0\relax%
    \decompose#1/\nil%
    \definitionspan{\texttt{\InlineClass{keyname}{\declare{#1}}}}%
    \defaultspan{(\extrakeytext initially \texttt{#2})}}%
}

% Introduces a key alias '/#1/<name of current key>'
% to be used inside of \begin{key} ... \end{key}
\def\keyalias#1{\vspace{-3pt}\item{\small alias \texttt{/#1/\myname}}\vspace{-2pt}\par
  \pgfmanualpdflabel{/#1/\myname}{}%
}

\newif\iffirsttime
\newcount\pgfmanualdecomposecount

\makeatother

\def\decompose/#1/#2\nil{%
  \def\test{#2}%
  \ifx\test\empty%
    % aha.
    \index{#1@\protect\texttt{#1} key}%
    \index{\mypath#1@\protect\texttt{#1}}%
    \gdef\myname{#1}%
    \pgfmanualpdflabel{#1}{}
  \else%
    \advance\pgfmanualdecomposecount by1\relax%
    \ifnum\pgfmanualdecomposecount>2\relax%
      \decomposetoodeep#1/#2\nil%
    \else%
      \iffirsttime%
        \begingroup%  
          % also make a pdf link anchor with full key path.
          \def\hyperlabelwithoutslash##1/\nil{%
            \pgfmanualpdflabel{##1}{}%
          }%
          \hyperlabelwithoutslash/#1/#2\nil%
        \endgroup%
        \def\mypath{#1@\protect\texttt{/#1/}!}%
        \firsttimefalse%
      \else%
        \expandafter\def\expandafter\mypath\expandafter{\mypath#1@\protect\texttt{#1/}!}%
      \fi%
      \def\firsttime{}%
      \decompose/#2\nil%
    \fi%
  \fi%
}

\def\decomposetoodeep#1/#2/\nil{%
  % avoid too-deep nesting in index
  \index{#1/#2@\protect\texttt{#1/#2} key}%
  \index{\mypath#1/#2@\protect\texttt{#1/#2}}%
  \decomposefindlast/#1/#2/\nil%
}
\makeatletter
\def\decomposefindlast/#1/#2\nil{%
  \def\test{#2}%
  \ifx\test\pgfutil@empty%
    \gdef\myname{#1}%
  \else%
    \decomposefindlast/#2\nil%
  \fi%
}
\makeatother
\def\indexkey#1{%
  \def\mypath{}%
  \decompose#1/\nil%
}

\newenvironment{predefinedmethod}[1]{
  \begin{pgfmanualentry}
    \extractpredefinedmethod#1\@nil
    \pgfmanualbody
}
{
  \end{pgfmanualentry}
}
\def\extractpredefinedmethod#1(#2)\@nil{%
  \pgfmanualentryheadline{%
    \pgfmanualpdflabel{#1}{}%
    \definitionspan{Method \declare\texttt{#1}\texttt(#2\texttt)} %
    \defaultspan{(predefined for all classes)}}
  \index{#1@\protect\texttt{#1} method}%
  \index{Methods!#1@\protect\texttt{#1}}
}


\newenvironment{ooclass}[1]{
  \begin{pgfmanualentry}
    \def\currentclass{#1}
    \pgfmanualentryheadline{%
      \pgfmanualpdflabel{#1}{}%
      \definitionspan{\textbf{Class} \declare{\texttt{#1}}}}
    \index{#1@\protect\texttt{#1} class}%
    \index{Class #1@Class \protect\texttt{#1}}%
    \index{Classes!#1@\protect\texttt{#1}}
    \pgfmanualbody
}
{
  \end{pgfmanualentry}
}

\newenvironment{method}[1]{
  \begin{pgfmanualentry}
    \extractmethod#1\@nil
    \pgfmanualbody
}
{
  \end{pgfmanualentry}
}
\def\extractmethod#1(#2)\@nil{%
  \def\test{#1}
  \ifx\test\currentclass
    \pgfmanualentryheadline{%
      \pgfmanualpdflabel{#1}{}%
      \definitionspan{Constructor \declare\texttt{#1}\texttt(#2\texttt)}}
  \else
    \pgfmanualentryheadline{%
      \pgfmanualpdflabel{#1}{}%
      \definitionspan{Method \declare\texttt{#1}\texttt(#2\texttt)}}
  \fi
  \index{#1@\protect\texttt{#1} method}%
  \index{Methods!#1@\protect\texttt{#1}}
  \index{Class \currentclass!#1@\protect\texttt{#1}}%
}

\newenvironment{classattribute}[1]{
  \begin{pgfmanualentry}
    \extractattribute#1\@nil
    \pgfmanualbody
}
{
  \end{pgfmanualentry}
}
\def\extractattribute#1=#2;\@nil{%
  \def\test{#2}%
  \ifx\test\@empty
    \pgfmanualentryheadline{%
      \pgfmanualpdflabel{#1}{}%
      \definitionspan{Private attribute \declare\texttt{#1}}%
      \defaultspan{(initially empty)}}
  \else
    \pgfmanualentryheadline{%
      \pgfmanualpdflabel{#1}{}%
      \definitionspan{Private attribute \declare\texttt{#1}}%
      \defaultspan{(initially \texttt{#2})}}
  \fi
  \index{#1@\protect\texttt{#1} attribute}%
  \index{Attributes!#1@\protect\texttt{#1}}
  \index{Class \currentclass!#1@\protect\texttt{#1}}%
}



\newenvironment{predefinednode}[1]{
  \begin{pgfmanualentry}
    \pgfmanualentryheadline{%
      \pgfmanualpdflabel{#1}{}%
      \definitionspan{\textbf{Predefined node} \texttt{\declare{#1}}}}%
    \index{#1@\protect\texttt{#1} node}%
    \index{Predefined node!#1@\protect\texttt{#1}}
    \pgfmanualbody
}
{
  \end{pgfmanualentry}
}

\newenvironment{coordinatesystem}[1]{
  \begin{pgfmanualentry}
    \pgfmanualentryheadline{%
      \pgfmanualpdflabel{#1}{}%
      \definitionspan{\textbf{Coordinate system} \texttt{\declare{#1}}}}%
    \index{#1@\protect\texttt{#1} coordinate system}%
    \index{Coordinate systems!#1@\protect\texttt{#1}}
    \pgfmanualbody
}
{
  \end{pgfmanualentry}
}

\newenvironment{snake}[1]{
  \begin{pgfmanualentry}
    \pgfmanualentryheadline{\definitionspan{\textbf{Snake} \texttt{\declare{#1}}}}%
    \index{#1@\protect\texttt{#1} snake}%
    \index{Snakes!#1@\protect\texttt{#1}}
    \pgfmanualbody
}
{
  \end{pgfmanualentry}
}

\newenvironment{decoration}[1]{
  \begin{pgfmanualentry}
    \pgfmanualentryheadline{\definitionspan{\textbf{Decoration} \texttt{\declare{#1}}}}%
    \index{#1@\protect\texttt{#1} decoration}%
    \index{Decorations!#1@\protect\texttt{#1}}
    \pgfmanualbody
}
{
  \end{pgfmanualentry}
}


\def\pgfmanualbar{\char`\|}
\makeatletter
\newenvironment{pathoperation}[3][]{
  \begin{pgfmanualentry}
    \def\pgfmanualtest{#1}%
    \pgfmanualentryheadline{%
      \ifx\pgfmanualtest\@empty%
        \pgfmanualpdflabel{#2}{}%
      \fi%
      \definitionspan{\textcolor{gray}{\texttt{\char`\\path}\
        \ \dots}
      \declare{\texttt{\noligs{#2}}}#3\ \textcolor{gray}{\dots\texttt{;}}}}%
    \ifx\pgfmanualtest\@empty%
      \index{#2@\protect\texttt{#2} path operation}%
      \index{Path operations!#2@\protect\texttt{#2}}%
    \fi%
    \pgfmanualbody
}
{
  \end{pgfmanualentry}
}
\newenvironment{datavisualizationoperation}[3][]{
  \begin{pgfmanualentry}
    \def\pgfmanualtest{#1}%
    \pgfmanualentryheadline{%
      \ifx\pgfmanualtest\@empty%
        \pgfmanualpdflabel{#2}{}%
      \fi%
      \definitionspan{\textcolor{gray}{\texttt{\char`\\datavisualization}\
        \ \dots}
      \declare{\texttt{\noligs{#2}}}#3\ \textcolor{gray}{\dots\texttt{;}}}}%
    \ifx\pgfmanualtest\@empty%
      \index{#2@\protect\texttt{#2} (data visualization)}%
      \index{Data visualization!#2@\protect\texttt{#2}}%
    \fi%
    \pgfmanualbody
}
{
  \end{pgfmanualentry}
}
\makeatother

\def\doublebs{\texttt{\char`\\\char`\\}}


\newenvironment{package}[1]{
  \begin{pgfmanualentry}
    \pgfmanualentryheadline{%
      \pgfmanualpdflabel{#1}{}%
      \definitionspan{\texttt{\char`\\usepackage\char`\{\declare{#1}\char`\}\space\space \char`\%\space\space  \LaTeX}}}
    \index{#1@\protect\texttt{#1} package}%
    \index{Packages and files!#1@\protect\texttt{#1}}%
    \pgfmanualentryheadline{\definitionspan{\texttt{\char`\\input \declare{#1}.tex\space\space\space \char`\%\space\space  plain \TeX}}}
    \pgfmanualentryheadline{\definitionspan{\texttt{\char`\\usemodule[\declare{#1}]\space\space \char`\%\space\space  Con\TeX t}}}
    \pgfmanualbody
}
{
  \end{pgfmanualentry}
}


\newenvironment{pgfmodule}[1]{
  \begin{pgfmanualentry}
    \pgfmanualentryheadline{%
      \pgfmanualpdflabel{#1}{}%
      \definitionspan{\texttt{\char`\\usepgfmodule\char`\{\declare{#1}\char`\}\space\space\space
        \char`\%\space\space  \LaTeX\space and plain \TeX\space and pure pgf}}}
    \index{#1@\protect\texttt{#1} module}%
    \index{Modules!#1@\protect\texttt{#1}}%
    \pgfmanualentryheadline{\definitionspan{\texttt{\char`\\usepgfmodule[\declare{#1}]\space\space \char`\%\space\space  Con\TeX t\space and pure pgf}}}
    \pgfmanualbody
}
{
  \end{pgfmanualentry}
}

\newenvironment{pgflibrary}[1]{
  \begin{pgfmanualentry}
    \pgfmanualentryheadline{%
      \pgfmanualpdflabel{#1}{}%
      \definitionspan{\textbf{\tikzname\ Library} \texttt{\declare{#1}}}}
    \index{#1@\protect\texttt{#1} library}%
    \index{Libraries!#1@\protect\texttt{#1}}%
    \vskip.25em%
    {\texttt{\char`\\usepgflibrary\char`\{\declare{#1}\char`\}\space\space\space
        \char`\%\space\space  \LaTeX\space and plain \TeX\space and pure pgf}}\\
    {\texttt{\char`\\usepgflibrary[\declare{#1}]\space\space \char`\%\space\space  Con\TeX t\space and pure pgf}}\\
    {\texttt{\char`\\usetikzlibrary\char`\{\declare{#1}\char`\}\space\space
        \char`\%\space\space  \LaTeX\space and plain \TeX\space when using \tikzname}}\\
    {\texttt{\char`\\usetikzlibrary[\declare{#1}]\space
        \char`\%\space\space  Con\TeX t\space when using \tikzname}}\\[.5em]
    \pgfmanualbody
}
{
  \end{pgfmanualentry}
}

\newenvironment{purepgflibrary}[1]{
  \begin{pgfmanualentry}
    \pgfmanualentryheadline{%
      \pgfmanualpdflabel{#1}{}%
      \definitionspan{\textbf{{\small PGF} Library} \texttt{\declare{#1}}}}
    \index{#1@\protect\texttt{#1} library}%
    \index{Libraries!#1@\protect\texttt{#1}}%
    \vskip.25em%
    {\texttt{\char`\\usepgflibrary\char`\{\declare{#1}\char`\}\space\space\space
        \char`\%\space\space  \LaTeX\space and plain \TeX}}\\
    {\texttt{\char`\\usepgflibrary[\declare{#1}]\space\space \char`\%\space\space  Con\TeX t}}\\[.5em]
    \pgfmanualbody
}
{
  \end{pgfmanualentry}
}

\newenvironment{tikzlibrary}[1]{
  \begin{pgfmanualentry}
    \pgfmanualentryheadline{%
      \pgfmanualpdflabel{#1}{}%
      \definitionspan{\textbf{\tikzname\ Library} \texttt{\declare{#1}}}}
    \index{#1@\protect\texttt{#1} library}%
    \index{Libraries!#1@\protect\texttt{#1}}%
    \vskip.25em%
    {\texttt{\char`\\usetikzlibrary\char`\{\declare{#1}\char`\}\space\space \char`\%\space\space  \LaTeX\space and plain \TeX}}\\
    {\texttt{\char`\\usetikzlibrary[\declare{#1}]\space \char`\%\space\space Con\TeX t}}\\[.5em]
    \pgfmanualbody
}
{
  \end{pgfmanualentry}
}

\newenvironment{pgfkeyslibrary}[1]{%
	\begin{pgfmanualentry}%
		\pgfmanualentryheadline{%
			\pgfmanualpdflabel{#1}{}%
			\textbf{pgfkeys Library} \texttt{\declare{#1}}}%
		\index{#1@\protect\texttt{#1} pgfkeys library}%
		\index{pgfkeys Libraries!#1@\protect\texttt{#1}}%
		\vskip.25em
		{\texttt{\char`\\usepgfkeyslibrary\char`\{\declare{#1}\char`\}\space\space \char`\%\space\space  \LaTeX\space and plain \TeX}}\\
		{\texttt{\char`\\usepgfkeyslibrary[\declare{#1}]\space \char`\%\space\space Con\TeX t}}\\[.5em]%
		\pgfmanualbody
	}%
	{%
	\end{pgfmanualentry}%
}

\newenvironment{filedescription}[1]{
  \begin{pgfmanualentry}
    \pgfmanualentryheadline{\definitionspan{File \texttt{\declare{#1}}}}%
    \index{#1@\protect\texttt{#1} file}%
    \index{Packages and files!#1@\protect\texttt{#1}}%
    \pgfmanualbody
}
{
  \end{pgfmanualentry}
}


\newenvironment{packageoption}[1]{
  \begin{pgfmanualentry}
    \pgfmanualentryheadline{\definitionspan{\texttt{\char`\\usepackage[\declare{#1}]\char`\{pgf\char`\}}}}
    \index{#1@\protect\texttt{#1} package option}%
    \index{Package options for \textsc{pgf}!#1@\protect\texttt{#1}}%
    \pgfmanualbody
}
{
  \end{pgfmanualentry}
}



\newcommand\opt[1]{\textcolor{black!50!green}{#1}}
\newcommand\ooarg[1]{\texttt{[}\meta{#1}\texttt{]}}

% \def\opt{\afterassignment\pgfmanualopt\let\next=}
% \def\pgfmanualopt{\ifx\next\bgroup\bgroup\color{black!50!green}\else{\color{black!50!green}\next}\fi}



\def\beamer{\textsc{beamer}}
\def\pdf{\textsc{pdf}}
\def\eps{\texttt{eps}}
\def\pgfname{\textsc{pgf}}
% \def\tikzname{Ti\emph{k}Z}
\def\tikzname{\InlineClass{tikzname}{TikZ}} % italicized k looks bad in search engines
\def\pstricks{\textsc{pstricks}}
\def\prosper{\textsc{prosper}}
\def\seminar{\textsc{seminar}}
\def\texpower{\textsc{texpower}}
\def\foils{\textsc{foils}}

{
  \makeatletter
  \global\let\myempty=\@empty
  \global\let\mygobble=\@gobble
  \catcode`\@=12
  \gdef\getridofats#1@#2\relax{%
    \def\getridtest{#2}%
    \ifx\getridtest\myempty%
      \expandafter\def\expandafter\strippedat\expandafter{\strippedat#1}
    \else%
      \expandafter\def\expandafter\strippedat\expandafter{\strippedat#1\protect\printanat}
      \getridofats#2\relax%
    \fi%
  }

  \gdef\removeats#1{%
    \let\strippedat\myempty%
    \edef\strippedtext{\stripcommand#1}%
    \expandafter\getridofats\strippedtext @\relax%
  }
  
  \gdef\stripcommand#1{\expandafter\mygobble\string#1}
}

\def\printanat{\char`\@}

\newcommand\declare[1]{\textcolor{red!75!black}{#1}}
% \def\declare{\afterassignment\pgfmanualdeclare\let\next=}
% \def\pgfmanualdeclare{\ifx\next\bgroup\bgroup\color{red!75!black}\else{\color{red!75!black}\next}\fi}


\let\textoken=\command
\let\endtextoken=\endcommand

\def\myprintocmmand#1{\texttt{\char`\\#1}}

\def\example{\par\smallskip\noindent\textit{Example: }}
\def\themeauthor{\par\smallskip\noindent\textit{Theme author: }}


\def\indexoption#1{%
  \index{#1@\protect\texttt{#1} option}%
  \index{Graphic options and styles!#1@\protect\texttt{#1}}%
}

\def\itemcalendaroption#1{\item \declare{\texttt{#1}}%
  \index{#1@\protect\texttt{#1} date test}%
  \index{Date tests!#1@\protect\texttt{#1}}%
}



\def\class#1{\list{}{\leftmargin=2em\itemindent-\leftmargin\def\makelabel##1{\hss##1}}%
\extractclass#1@\par\topsep=0pt}
\def\endclass{\endlist}
\def\extractclass#1#2@{%
\item{{\texttt{\char`\\documentclass}#2\texttt{\char`\{\declare{#1}\char`\}}}}%
  \index{#1@\protect\texttt{#1} class}%
  \index{Classes!#1@\protect\texttt{#1}}}

\def\partname{Part}

\makeatletter
\def\index@prologue{\section*{Index}\addcontentsline{toc}{section}{Index}
  This index only contains automatically generated entries. A good
  index should also contain carefully selected keywords. This index is
  not a good index.
  \bigskip
}
\c@IndexColumns=2
  \def\theindex{\@restonecoltrue
    \columnseprule \z@  \columnsep 29\p@
    \twocolumn[\index@prologue]%
       \parindent -30pt
       \columnsep 15pt
       \parskip 0pt plus 1pt
       \leftskip 30pt
       \rightskip 0pt plus 2cm
       \small
       \def\@idxitem{\par}%
    \let\item\@idxitem \ignorespaces}
  \def\endtheindex{\onecolumn}
\def\noindexing{\let\index=\@gobble}


\newenvironment{arrowtipsimple}[1]{
  \begin{pgfmanualentry}
    \pgfmanualentryheadline{\definitionspan{\textbf{Arrow Tip Kind} \texttt{#1}}}
    \index{#1@\protect\texttt{#1} arrow tip}%
    \index{Arrow tips!#1@\protect\texttt{#1}}%
    \def\currentarrowtype{#1}
    \pgfmanualbody}
{
  \end{pgfmanualentry}
}

\newenvironment{arrowtip}[4]{
  \begin{pgfmanualentry}
    \pgfmanualentryheadline{\definitionspan{\textbf{Arrow Tip Kind} \texttt{#1}}}
    \index{#1@\protect\texttt{#1} arrow tip}%
    \index{Arrow tips!#1@\protect\texttt{#1}}%
    \pgfmanualbody
    \def\currentarrowtype{#1}
    \begin{minipage}[t]{10.25cm}
      #2
    \end{minipage}\hskip5mm\begin{minipage}[t]{4.75cm}
      \leavevmode\vskip-2em
    \tikz{
      \draw [black!50,line width=5mm,-{#1[#3,color=black]}] (-4,0) -- (0,0);
      \foreach \action in {#4}
      { \expandafter\processaction\action\relax }
    }
    \end{minipage}\par\smallskip
  }
{
  \end{pgfmanualentry}
}

\newenvironment{arrowcap}[5]{
  \begin{pgfmanualentry}
    \pgfmanualentryheadline{\definitionspan{\textbf{Arrow Tip Kind} \texttt{#1}}}
    \index{#1@\protect\texttt{#1} arrow tip}%
    \index{Arrow tips!#1@\protect\texttt{#1}}%
    \pgfmanualbody
    \def\currentarrowtype{#1}
    \begin{minipage}[t]{10.25cm}
      #2
    \end{minipage}\hskip5mm\begin{minipage}[t]{4.75cm}
      \leavevmode\vskip-2em
    \tikz{
      \path [tips, line width=10mm,-{#1[#3,color=black]}] (-4,0) -- (0,0);
      \draw [line width=10mm,black!50] (-3,0) -- (#5,0);
      \foreach \action in {#4}
      { \expandafter\processaction\action\relax }
    }
    \end{minipage}\par\smallskip
  }
{
  \end{pgfmanualentry}
}

\newenvironment{pattern}[1]{
  \begin{pgfmanualentry}
    \pgfmanualentryheadline{\definitionspan{\textbf{Pattern} \texttt{#1}}}
    \index{#1@\protect\texttt{#1} pattern}%
    \index{Patterns!#1@\protect\texttt{#1}}%
    \pgfmanualbody
}
{
  \end{pgfmanualentry}
}

\def\processaction#1=#2\relax{
  \expandafter\let\expandafter\pgf@temp\csname manual@action@#1\endcsname
  \ifx\pgf@temp\relax\else
    \pgf@temp#2/0/\relax
  \fi
}
\def\manual@action@length#1/#2/#3\relax{%
  \draw [red,|<->|,semithick,xshift=#2] ([yshift=4pt]current bounding
  box.north -| -#1,0) coordinate (last length) -- node
  [above=-2pt] {\texttt{length}} ++(#1,0);
}
\def\manual@action@width#1/#2/#3\relax{%
  \draw [overlay, red,|<->|,semithick] (.5,-#1/2) -- node [below,sloped] {\texttt{width}} (.5,#1/2);
}
\def\manual@action@inset#1/#2/#3\relax{%
  \draw [red,|<->|,semithick,xshift=#2] ([yshift=-4pt]current bounding
  box.south -| last length) -- node [below] {\texttt{inset}} ++(#1,0);
}

\newenvironment{arrowexamples}
{\begin{tabular}{llll}
    \emph{Appearance of the below at line width} \nextcell
    \emph{0.4pt} \nextcell
    \emph{0.8pt} \nextcell
    \emph{1.6pt} \\
  }
{\end{tabular}\vskip-1em}

\newenvironment{arrowcapexamples}
{\begin{tabular}{lll}
    \emph{Appearance of the below at line width} \nextcell
     \emph{1ex} \nextcell
     \emph{1em} \\
  }
{\end{tabular}\vskip-1em}

\def\arrowcapexample#1[#2]{\def\temp{#1}\ifx\temp\pgfutil@empty\arrowcapexample@\currentarrowtype[{#2}]\else\arrowcapexample@#1[{#2}]\fi}
\def\arrowcapexample@#1[#2]{%
  {\sfcode`\.1000\small\texttt{#1[#2]}} \nextcell
  \kern-.5ex\tikz [baseline,>={#1[#2]}] \draw [line
  width=1ex,->] (0,.5ex) -- (2em,.5ex);  \nextcell
  \kern-.5em\tikz [baseline,>={#1[#2]}] \draw [line
  width=1em,->] (0,.5ex) -- (2em,.5ex);  \\
}

\def\arrowexample#1[#2]{\def\temp{#1}\ifx\temp\pgfutil@empty\arrowexample@\currentarrowtype[{#2}]\else\arrowexample@#1[{#2}]\fi}
\def\arrowexample@#1[#2]{%
  {\sfcode`\.1000\small\texttt{#1[#2]}} \nextcell
  \tikz [baseline,>={#1[#2]}] \draw [line
  width=0.4pt,->] (0,.5ex) -- (2em,.5ex); thin \nextcell
  \tikz [baseline,>={#1[#2]}] \draw [line
  width=0.8pt,->] (0,.5ex) -- (2em,.5ex); \textbf{thick} \nextcell
  \tikz [baseline,>={#1[#2]}] \draw [line
  width=1.6pt,->] (0,.5ex) -- (3em,.5ex); \\
}

\def\arrowexampledup[#1]{\arrowexample[{#1] \currentarrowtype[}]}
\def\arrowexampledupdot[#1]{\arrowexample[{#1] . \currentarrowtype[}]}

\def\arrowexampledouble#1[#2]{\def\temp{#1}\ifx\temp\pgfutil@empty\arrowexampledouble@\currentarrowtype[{#2}]\else\arrowexampledouble@#1[{#2}]\fi}
\def\arrowexampledouble@#1[#2]{%
  {\sfcode`\.1000\small\texttt{#1[#2]} on double line} \nextcell
  \tikz [baseline,>={#1[#2]}]
    \draw [double equal sign distance,line width=0.4pt,->] (0,.5ex) -- (2em,.5ex); thin \nextcell
  \tikz [baseline,>={#1[#2]}]
    \draw [double equal sign distance,line width=0.8pt,->] (0,.5ex) -- (2em,.5ex); \textbf{thick} \nextcell
  \tikz [baseline,>={#1[#2]}]
    \draw [double equal sign distance, line width=1.6pt,->] (0,.5ex) -- (3em,.5ex); \\
}



\newcommand\symarrow[1]{%
  \index{#1@\protect\texttt{#1} arrow tip}%
  \index{Arrow tips!#1@\protect\texttt{#1}}%
  \texttt{#1}\nextcell yields thick  
  \begin{tikzpicture}[arrows={#1-#1},thick,baseline]
    \useasboundingbox (-1mm,-0.5ex) rectangle (1.1cm,2ex);
    \fill [black!15] (1cm,-.5ex) rectangle (1.1cm,1.5ex) (-1mm,-.5ex) rectangle (0mm,1.5ex) ;
    \draw (0pt,.5ex) -- (1cm,.5ex);
  \end{tikzpicture} and thin
  \begin{tikzpicture}[arrows={#1-#1},thin,baseline]
    \useasboundingbox (-1mm,-0.5ex) rectangle (1.1cm,2ex);
    \fill [black!15] (1cm,-.5ex) rectangle (1.1cm,1.5ex) (-1mm,-.5ex) rectangle (0mm,1.5ex) ;
    \draw (0pt,.5ex) -- (1cm,.5ex);
  \end{tikzpicture}
}
\newcommand\symarrowdouble[1]{%
  \index{#1@\protect\texttt{#1} arrow tip}%
  \index{Arrow tips!#1@\protect\texttt{#1}}%
  \texttt{#1}\nextcell yields thick  
  \begin{tikzpicture}[arrows={#1-#1},thick,baseline]
    \useasboundingbox (-1mm,-0.5ex) rectangle (1.1cm,2ex);
    \fill [black!15] (1cm,-.5ex) rectangle (1.1cm,1.5ex) (-1mm,-.5ex) rectangle (0mm,1.5ex) ;
    \draw (0pt,.5ex) -- (1cm,.5ex);
  \end{tikzpicture}
  and thin
  \begin{tikzpicture}[arrows={#1-#1},thin,baseline]
    \useasboundingbox (-1mm,-0.5ex) rectangle (1.1cm,2ex);
    \fill [black!15] (1cm,-.5ex) rectangle (1.1cm,1.5ex) (-1mm,-.5ex) rectangle (0mm,1.5ex) ;
    \draw (0pt,.5ex) -- (1cm,.5ex);
  \end{tikzpicture}, double 
  \begin{tikzpicture}[arrows={#1-#1},thick,baseline]
    \useasboundingbox (-1mm,-0.5ex) rectangle (1.1cm,2ex);
    \fill [black!15] (1cm,-.5ex) rectangle (1.1cm,1.5ex) (-1mm,-.5ex) rectangle (0mm,1.5ex) ;
    \draw[double,double equal sign distance] (0pt,.5ex) -- (1cm,.5ex);
  \end{tikzpicture} and 
  \begin{tikzpicture}[arrows={#1-#1},thin,baseline]
    \useasboundingbox (-1mm,-0.5ex) rectangle (1.1cm,2ex);
    \fill [black!15] (1cm,-.5ex) rectangle (1.1cm,1.5ex) (-1mm,-.5ex) rectangle (0mm,1.5ex) ;
    \draw[double,double equal sign distance] (0pt,.5ex) -- (1cm,.5ex);
  \end{tikzpicture}
}

\newcommand\sarrow[2]{%
  \index{#1@\protect\texttt{#1} arrow tip}%
  \index{Arrow tips!#1@\protect\texttt{#1}}%
  \index{#2@\protect\texttt{#2} arrow tip}%
  \index{Arrow tips!#2@\protect\texttt{#2}}%
  \texttt{#1-#2}\nextcell yields thick  
  \begin{tikzpicture}[arrows={#1-#2},thick,baseline]
    \useasboundingbox (-1mm,-0.5ex) rectangle (1.1cm,2ex);
    \fill [black!15] (1cm,-.5ex) rectangle (1.1cm,1.5ex) (-1mm,-.5ex) rectangle (0mm,1.5ex) ;
    \draw (0pt,.5ex) -- (1cm,.5ex);
  \end{tikzpicture} and thin
  \begin{tikzpicture}[arrows={#1-#2},thin,baseline]
    \useasboundingbox (-1mm,-0.5ex) rectangle (1.1cm,2ex);
    \fill [black!15] (1cm,-.5ex) rectangle (1.1cm,1.5ex) (-1mm,-.5ex) rectangle (0mm,1.5ex) ;
    \draw (0pt,.5ex) -- (1cm,.5ex);
  \end{tikzpicture}
}

\newcommand\sarrowdouble[2]{%
  \index{#1@\protect\texttt{#1} arrow tip}%
  \index{Arrow tips!#1@\protect\texttt{#1}}%
  \index{#2@\protect\texttt{#2} arrow tip}%
  \index{Arrow tips!#2@\protect\texttt{#2}}%
  \texttt{#1-#2}\nextcell yields thick  
  \begin{tikzpicture}[arrows={#1-#2},thick,baseline]
    \useasboundingbox (-1mm,-0.5ex) rectangle (1.1cm,2ex);
    \fill [black!15] (1cm,-.5ex) rectangle (1.1cm,1.5ex) (-1mm,-.5ex) rectangle (0mm,1.5ex) ;
    \draw (0pt,.5ex) -- (1cm,.5ex);
  \end{tikzpicture} and thin
  \begin{tikzpicture}[arrows={#1-#2},thin,baseline]
    \useasboundingbox (-1mm,-0.5ex) rectangle (1.1cm,2ex);
    \fill [black!15] (1cm,-.5ex) rectangle (1.1cm,1.5ex) (-1mm,-.5ex) rectangle (0mm,1.5ex) ;
    \draw (0pt,.5ex) -- (1cm,.5ex);
  \end{tikzpicture}, double 
  \begin{tikzpicture}[arrows={#1-#2},thick,baseline]
    \useasboundingbox (-1mm,-0.5ex) rectangle (1.1cm,2ex);
    \fill [black!15] (1cm,-.5ex) rectangle (1.1cm,1.5ex) (-1mm,-.5ex) rectangle (0mm,1.5ex) ;
    \draw[double,double equal sign distance] (0pt,.5ex) -- (1cm,.5ex);
  \end{tikzpicture} and 
  \begin{tikzpicture}[arrows={#1-#2},thin,baseline]
    \useasboundingbox (-1mm,-0.5ex) rectangle (1.1cm,2ex);
    \fill [black!15] (1cm,-.5ex) rectangle (1.1cm,1.5ex) (-1mm,-.5ex) rectangle (0mm,1.5ex) ;
    \draw[double,double equal sign distance] (0pt,.5ex) -- (1cm,.5ex);
  \end{tikzpicture}
}

\newcommand\carrow[1]{%
  \index{#1@\protect\texttt{#1} arrow tip}%
  \index{Arrow tips!#1@\protect\texttt{#1}}%
  \texttt{#1}& yields for line width 1ex
  \begin{tikzpicture}[arrows={#1-#1},line width=1ex,baseline]
    \useasboundingbox (-1mm,-0.5ex) rectangle (1.6cm,2ex);
    \fill [black!15] (1.5cm,-.5ex) rectangle (1.6cm,1.5ex) (-1mm,-.5ex) rectangle (0mm,1.5ex) ;
    \draw (0pt,.5ex) -- (1.5cm,.5ex);
  \end{tikzpicture}
}
\def\myvbar{\char`\|}
\newcommand\plotmarkentry[1]{%
  \index{#1@\protect\texttt{#1} plot mark}%
  \index{Plot marks!#1@\protect\texttt{#1}}
  \texttt{\char`\\pgfuseplotmark\char`\{\declare{\noligs{#1}}\char`\}} \nextcell
  \tikz\draw[color=black!25] plot[mark=#1,mark options={fill=examplefill,draw=black}] coordinates{(0,0) (.5,0.2) (1,0) (1.5,0.2)};\\
}
\newcommand\plotmarkentrytikz[1]{%
  \index{#1@\protect\texttt{#1} plot mark}%
  \index{Plot marks!#1@\protect\texttt{#1}}
  \texttt{mark=\declare{\noligs{#1}}} \nextcell \tikz\draw[color=black!25]
  plot[mark=#1,mark options={fill=examplefill,draw=black}] 
    coordinates {(0,0) (.5,0.2) (1,0) (1.5,0.2)};\\
}



\ifx\scantokens\@undefined
  \PackageError{pgfmanual-macros}{You need to use extended latex
    (elatex) or (pdfelatex) to process this document}{}
\fi

\begingroup
\catcode`|=0
\catcode`[= 1
\catcode`]=2
\catcode`\{=12
\catcode `\}=12
\catcode`\\=12 |gdef|find@example#1\end{codeexample}[|endofcodeexample[#1]]
|endgroup

% define \returntospace.
%
% It should define NEWLINE as {}, spaces and tabs as \space.
\begingroup
\catcode`\^=7
\catcode`\^^M=13
\catcode`\^^I=13
\catcode`\ =13%
\gdef\returntospace{\catcode`\ =13\def {\space}\catcode`\^^I=13\def^^I{\space}}
\gdef\showreturn{\show^^M}
\endgroup

\begingroup
\catcode`\%=13
\catcode`\^^M=13
\gdef\commenthandler{\catcode`\%=13\def%{\@gobble@till@return}}
\gdef\@gobble@till@return#1^^M{}
\gdef\@gobble@till@return@ignore#1^^M{\ignorespaces}
\gdef\typesetcomment{\catcode`\%=13\def%{\@typeset@till@return}}
\gdef\@typeset@till@return#1^^M{{\def%{\char`\%}\textsl{\char`\%#1}}\par}
\endgroup

% Define tab-implementation functions
%   \codeexample@tabinit@replacementchars@
% and
%   \codeexample@tabinit@catcode@
%
% They should ONLY be used in case that tab replacement is active.
%
% This here is merely a preparation step.
%
% Idea:
% \codeexample@tabinit@catcode@ will make TAB active
% and
% \codeexample@tabinit@replacementchars@ will insert as many spaces as
% /codeexample/tabsize contains.
{
\catcode`\^^I=13
% ATTENTION: do NOT use tabs in these definitions!!
\gdef\codeexample@tabinit@replacementchars@{%
 \begingroup
 \count0=\pgfkeysvalueof{/codeexample/tabsize}\relax
 \toks0={}%
 \loop
 \ifnum\count0>0
  \advance\count0 by-1
  \toks0=\expandafter{\the\toks0\ }%
 \repeat
 \xdef\codeexample@tabinit@replacementchars@@{\the\toks0}%
 \endgroup
 \let^^I=\codeexample@tabinit@replacementchars@@
}%
\gdef\codeexample@tabinit@catcode@{\catcode`\^^I=13}%
}%

% Called after any options have been set. It assigns
%   \codeexample@tabinit@catcode
% and
%   \codeexample@tabinit@replacementchars
% which are used inside of 
%\begin{codeexample}
% ...
%\end{codeexample}
%
% \codeexample@tabinit@catcode  is either \relax or it makes tab
% active.
%
% \codeexample@tabinit@replacementchars is either \relax or it inserts
% a proper replacement sequence for tabs (as many spaces as
% configured)
\def\codeexample@tabinit{%
  \ifnum\pgfkeysvalueof{/codeexample/tabsize}=0\relax
    \let\codeexample@tabinit@replacementchars=\relax
    \let\codeexample@tabinit@catcode=\relax
  \else
    \let\codeexample@tabinit@catcode=\codeexample@tabinit@catcode@
    \let\codeexample@tabinit@replacementchars=\codeexample@tabinit@replacementchars@
  \fi
}

\newif\ifpgfmanualtikzsyntaxhilighting

\pgfqkeys{/codeexample}{%
  width/.code=  {\setlength\codeexamplewidth{#1}},
  graphic/.code=  {\colorlet{graphicbackground}{#1}},
  code/.code=  {\colorlet{codebackground}{#1}},
  execute code/.is if=code@execute,
  hidden/.is if=code@hidden,
  code only/.code=  {\pgfmanual@setup@codefalse\code@executefalse},
  setup code/.code=  {\pgfmanual@setup@codetrue\code@executefalse},
  multipage/.code=  {\code@executefalse\pgfmanual@multipage@codetrue},
  pre/.store in=\code@pre,
  post/.store in=\code@post,
  % Dominik: extra values for imagecodeexample
  imagewidth/.store in=\code@imagewidth,
  imagesource/.store in=\code@imagesource,
  % Dominik: allow to specify that an image should stay an svg and not be replaced by the converted png 
  % (e.g., necessary because weird things happen for images with `pattern=` during png conversion)
  svg/.is if=code@svg,
  % #1 is the *complete* environment contents as it shall be
  % typeset. In particular, the catcodes are NOT the normal ones.
  typeset listing/.code=  {#1},
  render instead/.store in=\code@render,
  vbox/.code=  {\def\code@pre{\vbox\bgroup\setlength{\hsize}{\linewidth-6pt}}\def\code@post{\egroup}},
  ignorespaces/.code=  {\let\@gobble@till@return=\@gobble@till@return@ignore},
  leave comments/.code=  {\def\code@catcode@hook{\catcode`\%=12}\let\commenthandler=\relax\let\typesetcomment=\relax},
  tabsize/.initial=0,% FIXME : this here is merely used for indentation. It is just a TAB REPLACEMENT.
  every codeexample/.style={width=4cm+7pt, tikz syntax=true},
  from file/.code={\codeexamplefromfiletrue\def\codeexamplesource{#1}},
  tikz syntax/.is if=pgfmanualtikzsyntaxhilighting,
  animation list/.store in=\code@animation@list,
  animation pre/.store in=\code@animation@pre,
  animation post/.store in=\code@animation@post,
  animation scale/.store in=\pgfmanualanimscale,
  animation bb/.style={
    animation pre={
      \tikzset{
        every picture/.style={
          execute at begin picture={
            \useasboundingbox[clip] #1;}
        }
      }
    }
  },
  preamble/.store in=\code@preamble,
}

\def\pgfmanualanimscale{.5}

\newread\examplesource


% Opening, reading and closing the results file

\def\opensource#1{
  \immediate\openin\examplesource=#1
}
\def\do@codeexamplefromfile{%
  \immediate\openin\examplesource\expandafter{\codeexamplesource}%
  \def\examplelines{}%
  \readexamplelines
  \closein\examplesource
  \expandafter\endofcodeexample\expandafter{\examplelines}%
}

\def\readexamplelines{
  \ifeof\examplesource%
  \else
    \immediate\read\examplesource to \exampleline
    \expandafter\expandafter\expandafter\def\expandafter\expandafter\expandafter\examplelines\expandafter\expandafter\expandafter{\expandafter\examplelines\exampleline}
    \expandafter\readexamplelines%
  \fi
}

\let\code@animation@pre\pgfutil@empty
\let\code@animation@post\pgfutil@empty
\let\code@animation@list\pgfutil@empty

\let\code@pre\pgfutil@empty
\let\code@post\pgfutil@empty
\let\code@imagewidth\pgfutil@empty
\let\code@imagesource\pgfutil@empty
\let\code@render\pgfutil@empty
\let\code@preamble\pgfutil@empty
\def\code@catcode@hook{}

\newif\ifpgfmanual@multipage@code
\newif\ifpgfmanual@setup@code
\newif\ifcodeexamplefromfile
\newdimen\codeexamplewidth
\newif\ifcode@execute
\newif\ifcode@hidden
\newif\ifcode@svg
\newbox\codeexamplebox
\def\codeexample[#1]{%
  \global\let\pgfmanual@do@this\relax%
  \aftergroup\pgfmanual@do@this%
  \begingroup%
  \code@executetrue%
  \pgfqkeys{/codeexample}{every codeexample,#1}%
  \pgfmanualswitchoncolors%
  \ifcodeexamplefromfile\begingroup\fi
  \codeexample@tabinit% assigns \codeexample@tabinit@[catcode,replacementchars]
  \parindent0pt
  \begingroup%
  \par% this \par is not inside \ifcode@hidden because we want to switch to vmode
  \ifcode@hidden\else
    \medskip%
  \fi
  \let\do\@makeother%
  \dospecials%
  \obeylines%
  \@vobeyspaces%
  \catcode`\%=13%
  \catcode`\^^M=13%
  \code@catcode@hook%
  \codeexample@tabinit@catcode
  \relax%
  \ifcodeexamplefromfile%
    \expandafter\do@codeexamplefromfile%
  \else%
    \expandafter\find@example%
  \fi}
\def\endofcodeexample#1{%
  \endgroup%
  \ifpgfmanual@setup@code%
    \gdef\pgfmanual@do@this{%
      {%
        \returntospace%
        \commenthandler%
        \xdef\code@temp{#1}% removes returns and comments
      }%
      \BlockClassSingle{hidden}{%
      \edef\pgfmanualmcatcode{\the\catcode`\^^M}%
      \catcode`\^^M=9\relax%
      \expandafter\scantokens\expandafter{\code@temp}%
      \catcode`\^^M=\pgfmanualmcatcode%
      }%
    }%
  \fi%
  \ifcode@hidden\else
  \begin{BlockClass}{example}%
    \ifx\code@imagesource\pgfutil@empty
      \ifcode@execute%
        % \begin{BlockClass}{example-image}%
        % if code@svg, class should be "example-image prefers-svg", otherwise "example-image"
        \begin{BlockClass}{example-image\ifcode@svg\space prefers-svg\fi}%
        \ifx\code@render\pgfutil@empty%
          {% 
            {%
              \returntospace%
              \commenthandler%
              \xdef\code@temp{#1}% removes returns and comments
            }%
            \edef\pgfmanualmcatcode{\the\catcode`\^^M}%
            \catcode`\^^M=9\relax%
            \code@pre\expandafter\scantokens\expandafter{\code@temp\ignorespaces}\code@post\ignorespaces
            % \expandafter\scantokens\expandafter{\code@temp}%
            \catcode`\^^M=\pgfmanualmcatcode%
          }%
        \fi
        \end{BlockClass}% % end example-image
      \fi%
    \else%
      \begin{BlockClass}{example-image}%
        \includegraphics{\code@imagesource}%
      \end{BlockClass}%
    \fi%
    \def\code@width{\linewidth-6pt}
    \pgfkeysvalueof{/codeexample/prettyprint/base color}%
    \begin{BlockClass}{example-code}%
          \pgfmanualdolisting{#1}%
    \end{BlockClass}% % end example-code
    \par%
    \medskip
  \end{BlockClass}% % end example
  \fi
  \endcodeexample\endgroup%
}

% \input{pgfmanual-en-macros-imagecode.tex}

\def\endcodeexample{\endgroup}
\newbox\codeexampleboxanim

\def\pgfmanualdolisting#1{%
      {%
        \let\do\@makeother
        \dospecials
        \frenchspacing\@vobeyspaces
        \normalfont\ttfamily\footnotesize
        \typesetcomment%
        \codeexample@tabinit@replacementchars
        \@tempswafalse
        \def\par{%
          \if@tempswa
          \leavevmode \null \@@par\penalty\interlinepenalty
          \else
          \@tempswatrue
          \ifhmode\@@par\penalty\interlinepenalty\fi
          \fi}%
        \def\par{\\}%
        \obeylines
        \everypar \expandafter{\the\everypar \unpenalty}%
        \ifx\code@preamble\pgfutil@empty\else
          \pgfutil@tempdima=\hsize
          \InlineClass{codepreamble}{%
            \vbox{\hsize=\pgfutil@tempdima
                \raggedright\scriptsize\detokenize\expandafter{\code@preamble}}%
          }%
        \fi
        \pgfkeysvalueof{/codeexample/typeset listing/.@cmd}{#1}\pgfeov
      }%
}

\makeatother

\usepackage{pgfmanual}

%DOMINIK: get rid of lange and rangle
\makeatletter
\renewcommand{\meta}[1]{%
    \InlineClass{angle}{\HTMLentity{langle}}%
     \ifmmode \expandafter \nfss@text \fi
     {%
      \edef\meta@hyphen@restore
        {\hyphenchar\the\font\the\hyphenchar\font}%
      \hyphenchar\font\m@ne
      \language\l@nohyphenation
      \textit{#1}\/%
      \meta@hyphen@restore
     }\InlineClass{angle}{\HTMLentity{rangle}}%
}
% \renewcommand{\meta}[1]{%
%     \InlineClass{angle}{\HTMLentity{langle}}%
%     \textit{#1}%
%     \InlineClass{angle}{\HTMLentity{rangle}}%
% }
\makeatother

% autoxref is now always on

% \makeatletter
% % \pgfautoxrefs will be defined by 'make dist'
% \pgfutil@ifundefined{pgfautoxrefs}{%
%   \renewcommand\pgfmanualpdflabel[3][]{#3}% NO-OP
%   \def\pgfmanualpdfref#1#2{#2}%
%   \pgfkeys{
%     /pdflinks/codeexample links=false,% DISABLED.
%   }%
% }{}
% \makeatother

\newdimen\pgfmanualcslinkpreskip

% Styling of the pretty printer
\pgfkeys{
  /codeexample/syntax hilighting/.style={
    /codeexample/prettyprint/key name/.code={\textcolor{keycolor}{\pgfmanualpdfref{##1}{\noligs{##1}}}},
    /codeexample/prettyprint/key name with handler/.code 2 args={\textcolor{keycolor}{\pgfmanualpdfref{##1}{\noligs{##1}}}/\textcolor{blue!70!black}{\pgfmanualpdfref{/handlers/##2}{\noligs{##2}}}},
    /codeexample/prettyprint/key value display only/.code={\textcolor{keycolor}{{\itshape{\let\pgfmanualwordstartup\relax\pgfmanualprettyprintcode{##1}}}}},
    /codeexample/prettyprint/cs/.code={\textcolor{cscolor}{\pgfmanualcslinkpreskip4.25pt\pgfmanualpdfref{##1}{\noligs{##1}}}},
    /codeexample/prettyprint/cs with args/.code 2 args={\textcolor{black}{\pgfmanualcslinkpreskip4.25pt\pgfmanualpdfref{##1}{\noligs{##1}}}\{\textcolor{black}{\pgfmanualprettyprintcode{##2}}\pgfmanualclosebrace},
    /codeexample/prettyprint/cs arguments/pgfkeys/.initial=1,
    /codeexample/prettyprint/cs/pgfkeys/.code 2 args={\textcolor{black}{\pgfmanualcslinkpreskip4.25pt\pgfmanualpdfref{##1}{\noligs{##1}}}\{\textcolor{black}{\pgfmanualprettyprintpgfkeys{##2}}\pgfmanualclosebrace},
    /codeexample/prettyprint/cs arguments/begin/.initial=1,
    /codeexample/prettyprint/cs/begin/.code 2 args={\textcolor{black}{##1}\{\textcolor{cscolor}{\pgfmanualpdfref{##2}{\noligs{##2}}}\pgfmanualclosebrace},
    /codeexample/prettyprint/cs arguments/end/.initial=1,
    /codeexample/prettyprint/cs/end/.code 2 args={\textcolor{black}{##1}\{\textcolor{cscolor}{\pgfmanualpdfref{##2}{\noligs{##2}}}\pgfmanualclosebrace},
    /codeexample/prettyprint/word/.code={\pgfmanualwordstartup{\begingroup\pgfkeyssetvalue{/pdflinks/search key prefixes in}{}\pgfmanualpdfref{##1}{\noligs{##1}}\endgroup}},
    /codeexample/prettyprint/point/.code={\textcolor{pointcolor}{\noligs{##1}}},%
    /codeexample/prettyprint/point with cs/.code 2 args={\textcolor{pointcolor}{(\pgfmanualpdfref{##1}{\noligs{##1}}:\noligs{##2}}},%
    /codeexample/prettyprint/comment font=\itshape,
    /codeexample/prettyprint/base color/.initial=\color{basecolor},
    /pdflinks/render hyperlink/.code={%
      {\setbox0=\hbox{##1}%
        \rlap{{\color{linkcolor}\dimen0\wd0\advance\dimen0by-\pgfmanualcslinkpreskip\hskip\pgfmanualcslinkpreskip\vrule width\dimen0 height-1pt depth1.6pt}}%
        \box0%
      }%
    }
  },/codeexample/syntax hilighting
}

\colorlet{keycolor}{black}
\colorlet{cscolor}{black}
\colorlet{pointcolor}{black}
\colorlet{basecolor}{black}
\colorlet{linkcolor}{black!8}

\def\pgfmanualswitchoncolors{%
  \colorlet{keycolor}{green!50!black}%
  \colorlet{cscolor}{blue!70!black}
  \colorlet{pointcolor}{violet}
  \colorlet{basecolor}{black!55}
  \colorlet{linkcolor}{white}
}

\makeatletter
% custom comment command for styling
\newenvironment{codecomment}{%
  \LWR@subhtmlelementclass{span}{example-code-comment}%
  \begin{LWR@nestspan}%
}{%
  \LWR@htmltagc{/span}%
  \end{LWR@nestspan}%
}

\begin{warpHTML}
% better label names (remove LWR-ht- prefix)
\RenewDocumentCommand{\LWR@hypertargetb}{m +m}{%
  \label{#1}%
  #2%
  \endgroup%
}
\RenewDocumentCommand{\LWR@hyperlinkb}{m}{%
\ifbool{LWR@insidemathcomment}%
{\endgroup}%
{\LWR@hyperrefcb[#1]}%
}
\end{warpHTML}
\makeatother

\makeatletter

\def\pgfmanualwordstartup{\textcolor{black}}

\def\noligs#1{\pgfmanualnoligs#1\kern0pt--\pgf@stop}%
\def\pgfmanualnoligs#1--{%
  \pgfutil@ifnextchar\pgf@stop{#1\pgfutil@gobble}{#1-\kern0pt-\kern0pt\pgfmanualnoligs}%
}
\makeatother


%%% Local Variables: 
%%% mode: latex
%%% TeX-master: "beameruserguide"
%%% End: 


% %%%%% MODIFY HTML FILE NAME SYSTEM
% lwarp to use short section names for filenames but not for TOC

\makeatletter
\begin{warpHTML}
\DeclareDocumentCommand{\LWR@section}{m m m m}{%
\IfValueTF{#2}%
    {\LWR@traceinfo{LWR@section: starting #4 #2}}%
    {\LWR@traceinfo{LWR@section: starting #4 #3}}%
    \LWR@spanwarninvalid{section}%
\LWR@maybeprintpendingfootnotes{\csuse{LWR@depth#4}}%
\LWR@stoppars%
\LWR@startnewdepth{#4}%
\global\boolfalse{LWR@minipagethispar}%
\LWR@traceinfo{LWR@section: testing whether to start a new HTML file}%
\IfBooleanT{#1}{\LWR@traceinfo{LWR@section: starred}}%
\ifbool{LWR@forcinghtmlpage}{\LWR@traceinfo{LWR@section: forcinghtmlpage}}{}%
\ifthenelse{%
    \(%
        \(\NOT\equal{#1}{\BooleanTrue}\)\OR%
        \(\cnttest{\@nameuse{LWR@depth#4}}{=}{\LWR@depthpart}\)\OR%
        \(\boolean{LWR@forcinghtmlpage}\)%
    \)%
    \AND%
    \cnttest{\@nameuse{LWR@depth#4}}{<=}{\value{FileDepth}}%
    \AND%
    \(%
        \NOT\boolean{CombineHigherDepths}\OR%
        \cnttest{\@nameuse{LWR@depth#4}}{<=}{\value{LWR@prevFileDepth}}%
    \)%
    \AND%
    \(% phantomsection
        \NOT\isempty{#3}%
        \OR%
        \(\NOT\equal{#1}{\BooleanTrue}\)%
    \)%
}%
{% new file
    \LWR@traceinfo{LWR@section: new HTML file}%
    \IfNoValueTF{#2}%
        {\LWR@newhtmlfile{#3}}%
        {\LWR@newhtmlfile{#2}}%
}% new file
{% not new file
    \LWR@traceinfo{LWR@section: not a new HTML file, about to LWR@orignewpage}%
    \LWR@maybe@orignewpage%
}% not new file

\IfValueT{#3}{%
    \LWR@traceinfo{LWR@section: about to LWR@setlatestname}%
    \LWR@setlatestname{#3}%
}%
\ifbool{HTMLDebugComments}{%
    \begingroup%
    \LWR@nullfonts%
    \IfBooleanTF{#1}% starred
    {\LWR@htmlcomment{Opening #4*}}%
    {%
            \LWR@htmlcomment{Opening #4 ``#3''}%
    }\LWR@orignewline%
    \endgroup%
}{}
\ifthenelse{%
    \cnttest{\@nameuse{LWR@depth#4}}{>=}{\LWR@depthparagraph}%
}%
    {\LWR@startpars}%
    {}%
\LWR@traceinfo{LWR@section: about to LWR@createautosec}%
\LWR@createautosec{\@nameuse{LWR@tag#4}}%
\setcounter{LWR@currentautosecpage}{\value{page}}%
\IfBooleanTF{#1}%
{%
    \LWR@traceinfo{LWR@section: starred}%
    \ifbool{LWR@forcinghtmltoc}%
    {%
        \addcontentsline{toc}{#4}{%
            \LWR@isolate{#3}%
        }%
    }%
    {}%
}% starred
{% not starred
    \ifthenelse{%
        \cnttest{\@nameuse{LWR@depth#4}}{<=}{\value{secnumdepth}}%
    }%
    {% if secnumdepth
        \LWR@traceinfo{LWR@section: about to test main matter}%
        \ifbool{LWR@mainmatter}%
        {%
            \LWR@traceinfo{LWR@section: yes mainmatter}%
            \refstepcounter{#4}%
            \LWR@traceinfo{LWR@section: about to addcontentsline}%
            \addcontentsline{toc}{#4}%
            {%
                \protect\numberline{%
                    \@nameuse{pre#4name}%
                    \@nameuse{the#4}%
                    \@nameuse{post#4name}%
                }%
                {%
                    \ignorespaces%
                    \LWR@isolate{#3}\protect\relax%
                }%
            }%
            \LWR@traceinfo{LWR@section: finished addcontentsline}%
        }% end of if main matter
        {% not main matter
            \LWR@traceinfo{LWR@section: no main matter}%
            \addcontentsline{toc}{#4}{%
                \LWR@isolate{#3}%
            }%
        }% end of not main matter
    }% end of secnumdepth
    {%
        \addcontentsline{toc}{#4}{%
            \LWR@isolate{#3}%
        }%
    }%
    \ifbool{LWR@mainmatter}%
    {%
        \ifthenelse{%
            \(\cnttest{\@nameuse{LWR@depth#4}}{<=}%
                {\value{secnumdepth}}\) \AND%
            \(\cnttest{\@nameuse{LWR@depth#4}}{=}{\LWR@depthpart}\)%
        }%
            {\@partnameformat}%
            {}%
        \LWR@traceinfo{LWR@section: about to print section number}%
        \ifthenelse{%
            \cnttest{\@nameuse{LWR@depth#4}}{<=}{\value{secnumdepth}}%
        }%
            {%
                \ifstrequal{#4}{part}%
                {\protect\LWR@sectionnumber{\@partcntformat{#4}}}%
                {%
                    \ifstrequal{#4}{chapter}%
                        {%
                            \LWR@printchaptername%
                            \protect\LWR@sectionnumber{\@chapcntformat{#4}}%
                        }%
                        {\protect\LWR@sectionnumber{\@seccntformat{#4}}}%
                }%
            }%
            {}%
        \LWR@traceinfo{LWR@section: finished print section number}%
    }{}%
}% not starred
\LWR@traceinfo{LWR@section: about to print the section name}%
\LWR@isolate{#3}%
\LWR@traceinfo{LWR@section: about to close the heading tag}%
\LWR@htmltag{\@nameuse{LWR@tag#4end}}%
\LWR@orignewline%
\LWR@traceinfo{LWR@section: about to create the LaTeX label}%
\setcounter{LWR@currentautosecfloatpage}{\value{page}}%
\LWR@newautopagelabel{LWR@currentautosecpage}\LWR@orignewline%
\ifbool{LWR@setseqfilelabel}{}{%
    \label{\BaseJobname-autofile-\arabic{LWR@htmlseqfilenumber}}%
    \booltrue{LWR@setseqfilelabel}%
}%
\ifthenelse{%
    \cnttest{\@nameuse{LWR@depth#4}}{<}{\LWR@depthparagraph}%
}%
    {\LWR@startpars}%
    {}%
\ifthenelse{%
    \NOT\equal{#1}{\BooleanTrue}\OR%
    \cnttest{\@nameuse{LWR@depth#4}}{=}{\LWR@depthpart}%
}%
    {% not starred
        \setcounter{LWR@prevFileDepth}{\@nameuse{LWR@depth#4}}%
    }% not starred
    {}%
\ifstrempty{#3}%
    {}%
    {%
        \global\boolfalse{LWR@forcinghtmlpage}%
        \global\boolfalse{LWR@forcinghtmltoc}%
    }%
\LWR@traceinfo{LWR@section: done}%
}
\end{warpHTML}
\makeatother

% remove weird space in table of content behind section number
\makeatletter
% original: \def\@seccntformat#1{\csname the#1\endcsname\quad}
\makeatother
